\section{Аналитический раздел}

В данном разделе формализуется задача, приводится требуемая функциональность разрабатываемого приложения, проводится анализ существующих решений.

\subsection{Формализация задачи}
Необходимо реализовать серверное приложение для доступа к базе данных, предназначенной для хранения информации о действиях и характеристиках, необходимых для определения усталости пользователей АРМ. Данная потребность связана с тем, что при работе с InfluxDB на ЯП Kotlin на текущий момент отсутствуют библиотеки, отвечающие полноте функционала, который может понадобится при проводимых работах.

В качестве решения поставленной задачи поставщики СУБД предлагают разработчику реализовать собственное серверное приложение, которое будет обрабатывать запросы, используя обращения к API развёрнутой СУБД.

К возможностям, которые должен предоставлять сервер, отнесены:
\begin{itemize}[leftmargin=1.6\parindent]
\item внесение данных;
\item получение данных;
\item проверка существования хранилища для пользователя;
\item создание хранилищ для новых пользователей.
\end{itemize}

\subsection{Существующие решения}
\subsubsection{The Kotlin InfluxDB 2.0 Client}
The Kotlin InfluxDB 2.0 Client \cite{influxClient} - это клиент, который предоставляет возможность производить запросы и запись в InfluxDB 2.0 с использованием ЯП Kotlin. Данная библиотека поддерживает асинхронные запросы с использованием Kotlin Coroutines.

На данный момент решение поддерживает следующий функционал:
\begin{itemize}[leftmargin=1.6\parindent]
\item запись в базу данных;
\item чтение базы данных с использованием стандартного языка InfluxQL;
\item чтение базы данных с использованием языка Flux.
\end{itemize}

Данным решением не поддерживается следующий требуемый функционал:
\begin{itemize}[leftmargin=1.6\parindent]
\item проверка существования хранилища для пользователя;
\item создание хранилищ для новых пользователей.
\end{itemize}

\subsubsection{InfluxDB v2 API}
InfluxDB поддерживает обращение к InfluxDB v2 API \cite{influxApi}. InfluxDB API предоставляет способ взаимодействия с базой данных с использованием HTTP-запросов и ответов, включающих в своё тело данные в формате JSON, HTTP аутентификации, а также с поддержкой токенов JWT и базовой аутентификации.

Предоставляемый данным интерфейсом функционал полон и непосредственно используется в реализации Web-клиента данной СУБД. К недостатку использования данного метода взаимодействия относятся формирование множественных HTTP-запросов и потребность в обработке ответов.

\subsubsection*{Вывод}
Среди рассмотренных существующих решений отсутствуют примеры удобной реализации, отвечающей полноте функционала, которую можно было бы использовать при написании приложений на ЯП Kotlin.

\subsection{Понятие клиент-серверного приложения}

\subsection{HTTP}



\subsubsection*{Вывод}
bluh-bluh-bluh, только после завершения потребуется написать, очень сильно хочу себе порше макан гтс, чтобы просто намотаться на столб. Выживать не обязательно. Спасибо.


\pagebreak