\section*{ЗАКЛЮЧЕНИЕ}
\addcontentsline{toc}{section}{ЗАКЛЮЧЕНИЕ}

Во время выполнения курсового проекта были достигнуты поставленные задачи:
\begin{itemize}
\item формализована задача,
\item определена требуемая функциональность,
\item проанализированы существующие решения,
\item описан протокол взаимодействия клиента и сервера,
\item разработано приложения для решения поставленной цели.
\end{itemize}

Проведённая аналитическая работа позволила формализовать задачу и определить требуемую функциональность, реализованную на сервере. Также были проанализированы существующие решения The Kotlin InfluxDB 2.0 Client и InfluxDB v2 API, которые в дальнейшем позволили реализовать приложение с использованием возможностей, предоставляемых данными решениями. Были рассмотрены понятия клиент-серверной архитектуры, шаблон проектирования Model-View-Controller и протокол HTTP, на основе которого был реализован собственный протокол взаимодействия сервера и пользователя YDVP.

В результате работы, проведенной в конструкторском разделе, были приведены диаграммы вариантов использования для клиента и репозитория. Также была определена схема работы сервера и описание протокола YDVP версии 0.1.

Для реализации в качестве используемого языка программирования был выбран ЯП Kotlin, а в качестве среды разработки -- IntelliJ IDEA.

В результате работы было реализовано приложение, состоящее из 5 модулей, структура и состав классов которых были представлены в графической интерпретации диаграммы классов. Приведены особенности реализации и пример использования приложения.

В ходе выполнения поставленных задач были получены знания в области компьютерных сетей, а также изучены возможности ЯП Kotlin, библиотеки OkHttp3, а также особенности работы сокетов на Java Virtual Machine.

В качестве дальнейшего развития проекта могут быть предприняты следующие действия:
\begin{itemize}
\item расширение стандарта YDVP и переход на новую версию,
\item увеличение доступных на сервере ресурсов,
\item переход на использование фреймворка SpringBoot,
\item добавление системы аутентификации,
\item добавление поддержки JWT-авторизации.
\end{itemize}

\pagebreak